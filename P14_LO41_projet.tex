\documentclass[12pt]{article}

\usepackage[utf8]{inputenc}
\usepackage[T1]{fontenc}
\usepackage[francais]{babel}
\usepackage{graphicx}
\usepackage{lscape}

\title{\textbf{Projet de LO41 : \\Gestion des déchets}}
\author{Simon MAGNIN-FEYSOT\\
		Aurélie PELLIGAND}
\date{\today}
%Responsable de l'UV : Philippe DESCAMPS
%Printemps 2014

%%%%%%%%Début du document %%%%%%
\begin{document}
\begin{titlepage}
\maketitle
\begin{abstract}Ce rapport illustre notre projet de l'UV LO41 concernant la gestion des déchets dans une commune. Il comporte des précisions générales sur l'ensemble du projet. Le projet consiste à définir et simuler le comportement de tous les acteurs (usagers, type de déchets, poubelles, centre de tri,service de ramassage, etc.) et de montrer leurs interactions.
\end{abstract}

\tableofcontents
\thispagestyle{empty}
\end{titlepage}

%%%%%%%%%%%% INTRODUCTION %%%%%%%%%%%%%%%
\section{Introduction}
\setcounter{page}{2}
Dans le cadre de l'UV LO41, intitulée Architecture et utilisation des systèmes d'exploitation, nous avons été amenés à développer ce projet permettant de mettre en avant nos compétences acquises au cours du semestre. \\

Les notions de processus, de synchronisation interprocessus, de gestion de ressources, des files de messages, mémoires partagée, des sémaphores, des threads, des mutex, des moniteurs et des signaux vont être utilisées dans notre projet.\\

Ce projet concerne la réalisation d'un gestionnaire de déchets dans une commune. Afin de partir sur de bonnes bases, nous avons conçu un réseau de pétri pour bien comprendre les intéractions entre les différents acteurs, ce qui permet une meilleure compréhension pour la programmation en C du projet.

%%%%%%%%%%%%%%%%%%%%%%%%%%%%%%%%%%%%%%%%%%%%%
\section{Enoncé}
\paragraph{}Alors que le coût de la gestion des déchets augmente dans toutes les collectivités, certaines d'entre elles font rimer optimisation de leur dépenses avec innovation technologiques. Dans le cadre de la levée individuelle, chaque poubelle est dotée d'une puce qui évalue, pour chaque foyer, les quantités de déchets générés via un système de facturation expérimentale comptabilisant le nombre de fois où le vac est vidé.\\

Pour ceux qui privilégient le dépôt à la borne collective, ils se voient alors remettre une clé qui leur permettra d'ouvrir le container. Dans la pratique, l’usager peut choisir entre 3 modes : (1) bac seul ; (2) clé seule ; (3) bac et clé. Si l’usager choisit la clé, il devra utiliser obligatoirement des sacs de 30 litres (capacité maximum des bornes). Si l’usager choisit le bac, sa taille dépendra de la composition du foyer selon le principe suivant :
\begin{itemize}
\item 1 personne: 80 litres
\item 2 personnes: 120 litres
\item 3 et 4 personnes: 180 litres
\item 5 personnes et plus: 240 litres
\end{itemize}

\paragraph{}Des points tri sont répartis sur la commune. Pour le papier, le verre et les emballages ménagers : l’accès à la borne est libre. Pour déposer les ordures ménagères dans les bornes à tambour, l’usager peut s ‘équiper d’une clé. Dans le cas où l’utilisateur ne peut pas suivre le calendrier des collectes, il peut bénéficier d’un service en apport volontaire, à la place ou en plus du bac individuel.
\paragraph{}Les poubelles sont dotées de capteurs facilitant le pesage et d’un système qui communique avec le service de ramassage. Cette solution facilite l’identification des points de collectes et en conséquence identifie les centres de collectes les plus critiques. Il appartiendra donc à la mairie de s’adapter aux conditions de collecte en fonction des périodes de l’année.
\paragraph{}Le projet consiste donc à définir et simuler le comportement de tous les acteurs (usagers, type de déchets, poubelles, centre de tri,service de ramassage, etc.) et de montrer leurs interactions.

%%%%%%%%%%%%%%%%%%%%%%%%%%%%%%%%%%%%%%%%%%%%%
\section{Analyse}
\subsection{Réseaux de pétri}
\paragraph{} Pour permettre une meilleure visualisation du programme, nous avons conçu un réseau de Petri. Cet outil graphique permet de modéliser le comportement dynamique de la gestion des poubelles.

\paragraph{}Réseaux de Petri pour un utilisateur : 
\begin{enumerate}
\item Création utilisateur 
\item L'utilisateur souhaite jeter sa poubelle
\item Descend vers la poubelle voulue (menager, verre, carton, collective)
\item L'utilisateur remplie la poubelle voulue si le remplissage de la poubelle de type ménager, verre, carton ou collective est inférieur à 70\% et si l'utilisateur a la clé ou clé bac pour les poubelles de type ménager.
\end{enumerate}

\begin{landscape}
\begin{figure}[h]
	\begin{center}
		\includegraphics[scale=0.65]{reseauxPetri_1user.PNG}
	\end{center}
	\caption{Réseaux de Petri des usagers}
\label{Petri}
\end{figure}
\end{landscape}
\paragraph{Explication du réseau de Petri utilisateur} Dans l'exemple de la figure 1, on créé un seul utilisateur. Il souhaite vider sa poubelle dans sa poubelle personnelle donc le point se déplace. L'utilisateur descend vers sa poubelle personnelle. On voit que le remplissage de la poubelle personnelle de l'utilisateur est inférieur à 70\% donc il peut jeter sa poubelle dans sa poubelle personnelle. La poubelle est alors déposée dans sa propre poubelle. Suite à ceci, soit la poubelle est toujours inférieur à 70\% dans ce cas le point se replace dans la place ("remplissage poubelle personnelle < 70\%") et un autre point se déplace dans la place ("l'utilisateur souhaite vider une poubelle"). 
\paragraph{}Voici le réseau de Petri qui explique comment se déroule le ramassage des poubelles par les camions. 
\begin{figure}[h]
	\begin{center}
		\includegraphics[scale=0.8]{reseauxPetri_1camion.PNG}
	\end{center}
	\caption{Réseaux de Petri des camions}
\label{Petri}
\end{figure}
\paragraph{}Lorsque la poubelle est remplie à plus de 70\%, la puce de la poubelle envoie un signal au centre de tri qui envoie un camion adapté à la poubelle. N'importe quelle poubelle remplie à plus de 70\% sera considérée comme remplie pour les camions.
\paragraph{}Nous n'avons pas réaliser de réseau de Petri global car tous les liens entre les utilisateurs, les poubelles et les camions seraient trop compliqués à comprendre sur un schéma. C'est pourquoi nous allons juste expliquer comme se passe ces étapes. 

\subsection{Installation et utilisation du programme}Ce programme est compatible avec linux (Debian 7 Wheeezy et Xubuntu 13.10). Pour compiler notre programme, il suffit de faire :\\ "gcc main.c -lpthread -g -o Poubelle".
\paragraph{}Pour exécuter notre programme, il faut lancer l'exécutable avec les 5 arguments suivants: \\./Poubelle NBUSAGERS NBCAMIONS NBPOUBELLESCOLLECTIVES NBPOUBELLEVERRE NBPOUBELLECARTON

\paragraph{}Les arguments sont obligatoires pour lancer le programme. La simulation se fait sur un nombre de jours prédéfini dans le programme.


%%%%%%%%%%%% conclusion %%%%%%%%%%%%%%%
\section{Conclusion}
\paragraph{}Ce projet a été fortement intéressant cela a été l'occasion de pouvoir mettre en pratique sur un cas concret les concepts vu tout au long du semestre : les threads, les moniteurs, les mutex, les signals... Nous avons su respecter le cahier des charges donné par l'énoncé. La solution que nous avons présentée nous paraît adéquate par rapport au problème posé. Nous avons mis en place un travail de groupe efficace grâce au partage du code des fichiers via Git.
\paragraph{}Nous aurions pu nous amuser à faire un algortihme de djikstra pour optimiser le voyage du camion poubelles que nous avons vu en AG41 (optimisation et recherche opérationnelle).

\end{document}

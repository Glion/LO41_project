\documentclass[12pt]{article}

\usepackage[utf8]{inputenc}  
\usepackage[T1]{fontenc}
\usepackage[francais]{babel}
\usepackage{graphicx}

\title{\textbf{Projet de LO41 : \\Gestion des déchets}}
\author{Simon MAGNIN-FEYSOT\\
		Aurélie PELLIGAND}
\date{\today}

%%%%%%%%Début du document %%%%%%
\begin{document}
\maketitle
\begin{abstract}
Voici le rapport du projet de LO41 sur la gestion des déchets.
\end{abstract}

%%%%%%%%%%%% INTRODUCTION %%%%%%%%%%%%%%%

\section{Introduction}
\paragraph{}Alors que le coût de la gestion des déchets augmente dans toutes les collectivités, certaines d'entre elles font rimer optimisation de leur dépenses avec innovation technologiques. Dans le cadre de la levée individuelle, chaque poubelle est dotée d'une puce qui évalue, pour chaque foyer, les quantités de déchets générés via un système de facturation expérimentale comptabilisant le nombre de fois où le vac est vidé.Pour ceux qui privilégient le dépôt à la borne collective, ils se voient alors remettre une clé qui leur permettra d'ouvrir le container. Dans la pratique, l’usager peut choisir entre 3 modes : (1) bac seul ; (2) clé seule ; (3) bac et clé. Si l’usager choisit la clé, il devra utiliser obligatoirement des sacs de 30 litres (capacité maximum des bornes). Si l’usager choisit le bac, sa taille dépendra de la composition du foyer selon le principe suivant :
\begin{itemize}
\item 1 personne: 80 litres
\item 2 personnes: 120 litres
\item 3 et 4 personnes: 180 litres
\item 5 personnes et plus: 240 litres
\end{itemize}
Des points tri sont répartis sur la commune. Pour le papier, le verre et les emballages ménagers : l’accès à la borne est libre. Pour déposer les ordures ménagères dans les bornes à tambour, l’usager peut s ‘équiper d’une clé. Dans le cas où l’utilisateur ne peut pas suivre le calendrier des collectes, il peut bénéficier d’un service en apport volontaire, à la place ou en plus du bac individuel. 

\paragraph{}Les poubelles sont dotées de capteurs facilitant le pesage et d’un système qui communique avec le service de ramassage. Cette solution facilite l’identification des points de collectes et en conséquence identifie les centres de collectes les plus critiques. Il appartiendra donc à la mairie de s’adapter aux conditions de collecte en fonction des périodes de l’année.

\paragraph{}Le projet consiste à définir et simuler le comportement tous les acteurs (usagers, type de déchets, poubelles, centre de tri,service de ramassage, etc.) et de montrer leurs interactions.

%%%%%%%%%%%%%%%%%%%%%%%%%%%%%%%%%%%%%%%%%%%%%
\section{Analyse}
\paragraph{}


%%%%%%%%%%%%%%%%%%%%%%%%%%%%%%%%%%%%%%%%%%%%%
\section{Réseau de Pétrie}

%%%%%%%%%%%%%%%%%%%%%%%%%%%%%%%%%%%%%%%%%%%%%
\section{Scenario test}


%%%%%%%%%%%% conclusion %%%%%%%%%%%%%%%
\section{Conclusion}
\paragraph{Avis sur ce qui serait bien de faire}


\paragraph{}Si nous avions eu plus de temps, il aurait fallu ...

\end{document}
